% !TeX encoding = UTF-8
% !TEX TS-program = latexmk -xelatex -shell-escape -silent -latexoption="-synctex=1 -8bit" %
%
% ^^^ This is the build command. Install latexmk if you don't have it already.
%     You may choose an alternative LaTeX derivative, e.g. LuaLaTeX, but be warned that it must support Unicode!

\documentclass{article}

\usepackage{fontspec}

% Download GNU Unifont from http://unifoundry.com/unifont/index.html
% And save it, say, as "fonts/unifont-13.0.03.ttf"
\newfontfamily\unifont{unifont-13.0.03.ttf}[Path=./fonts/,NFSSFamily=unifont]

% Disable caching for debugging purposes (increases compilation times!)
\usepackage[cache=false]{minted}
\setminted{fontfamily=unifont,tabsize=2,breaklines=true}

\newminted[mmtcode]{mmt}{}
\newmintinline[mmtinline]{mmt}{}
\newmintedfile[mmtfile]{mmt}{}

\begin{document}
	\section{Display Code}
	Use the verbatim environment \verb|\begin{mmtcode}...\end{mmtcode}| to produce display code blocks:
	\begin{mmtcode}
theory MyTheory =
	ℕ : type ❙
	succ : ℕ ⟶ℕ ❘ # S 1❙
❚
	\end{mmtcode}

	Using \verb|\mmtfile{filename.mmt}|, you can also embed display code from files:
	\mmtfile{test.mmt}

	\section{Inline Code}
	Use the inline verbatim command \verb|\mmtinline/.../| to produce inline code: \mmtinline/theory MyTheory = ℕ : type ❙ succ : ℕ ⟶ℕ ❘ # S 1❙❚/
\end{document}